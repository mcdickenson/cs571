\documentclass[landscape, paperwidth=42in, paperheight=36in,
fontscale=.35, margin=1in]{baposter} 

% Packages to use
\usepackage{calc}
\usepackage{graphicx}
\usepackage{amsmath}
\usepackage{amssymb}
\usepackage{relsize}
\usepackage{multirow}
\usepackage{rotating}
\usepackage{bm}
\usepackage{enumitem}
\usepackage{url}
\usepackage{booktabs}
\usepackage{graphicx}
\usepackage{multicol}
\usepackage{tikz}
\usepackage[longnamesfirst]{natbib}
\usepackage{array}
\usepackage[T1]{fontenc}
\usepackage[urw-garamond]{mathdesign}
\usepackage{garamondx}

% Define colors
\definecolor{lightblue}{rgb}{0.145,0.6666,1}
\definecolor{darkblue}{rgb}{0.220,0.424,0.690}
\definecolor{lightred}{rgb}{0.81,0.2,.13}
\definecolor{darkred}{rgb}{0.8,0.18,0.15}
\definecolor{black}{rgb}{0,0,0}
\definecolor{dukeblue}{rgb}{0,0,0.612}

% Define custom commands
\newcommand{\compresslist}{%
\setlength{\itemsep}{1pt}%
\setlength{\parskip}{0pt}%
\setlength{\parsep}{0pt}%
}

% Begin poster and set options
\begin{document}
\begin{poster}{ 
  % Show grid to help with alignment
  grid=false,
  columns=3,
  % Column spacing
  colspacing=0.1in, 
  % Color style
  bgColorOne=white,
  bgColorTwo=white,
  borderColor=black,
  headerColorOne=dukeblue, % was black (gradient)
  headerColorTwo=dukeblue, % lightblue
  headerFontColor=white,
  boxColorOne=white,
  boxColorTwo=black, % lightblue
  % Format of textbox
  textborder=roundedleft,
  % Format of text header
  eyecatcher=true,
  headerborder=closed,
  headerheight=0.1\textheight,
%  textfont=\sc, An example of changing the text font
% headershape can be one of: rectangle, rounded, smallrounded, roundedleft, roundedright
  headershape=rounded, 
  headershade=shadelr,
  headerfont=\Large\bf\textsf, %Sans Serif 
  textfont={\setlength{\parindent}{1.5em}},
  boxshade=plain,
%  background=shade-tb,
  background=plain,
  linewidth=2pt
  } {%%% Eye Catcher %%%%%%%%
	% \hspace{1in} 
 %        \setlength\fboxsep{0pt}
 %        \setlength\fboxrule{0.5pt}
 %        \fbox{\includegraphics[scale=1]{bric_countries.jpg}}
} {%%% Title %%%%%%%%
\textsf{
	\bf{Automated Production of Political Indicators}}
} { %%% Authors %%%%%%%
	\vspace{1em} \textsf{Matthew C. Dickenson}\\
	{\smaller \textsf{mcd31@duke.edu}}
} {%%% Logo %%%%%%%%%%
  \centering \includegraphics[scale=.8]{dukelogo_vert_black.eps} \hspace{1in}
}


%%%%%%%%%%%%%%%%%%%%%%%%%%%%%%%%%%%%%%%%%%%%%%%%%%%%%%%%%%%%%%%%%%%%%%%%%%%%%%
%%% Now define the boxes that make up the poster
%%%---------------------------------------------------------------------------
%%% Each box has a name and can be placed absolutely or relatively.
%%% The only inconvenience is that you can only specify a relative position 
%%% towards an already declared box. So if you have a box attached to the 
%%% bottom, one to the top and a third one which should be in between, you 
%%% have to specify the top and bottom boxes before you specify the middle 
%%% box.
%%%%%%%%%%%%%%%%%%%%%%%%%%%%%%%%%%%%%%%%%%%%%%%%%%%%%%%%%%%%%%%%%%%%%%%%%%%%%%
    %
    % A coloured circle useful as a bullet with an adjustably strong filling
    \newcommand{\colouredcircle}{%
      \tikz{\useasboundingbox (-0.2em,-0.32em) rectangle(0.2em,0.32em); \draw[draw=black,fill=lightblue,line width=0.03em] (0,0) circle(0.18em);}}

%%%%%%
  \headerbox{\bf Motivation}{name=question,column=0,row=0}{
%%%%%
Can political indicators such as the Militarized Interstate Dispute (MID) dataset be approximated by automated classification procedures? Existing, human-intensive pipelines for production of MID and related datasets (e.g. Polity and Freedom House) are costly and slow. This project uses classification trees to estimate the occurrence of MIDs using event data, described below. 
   \vspace{0.3em}
 }

%%%%%%
  \headerbox{\bf Data Source}{name=data,below=question,above=bottom}{
%%%%%
  Features were drawn from the Global Database of Events, Language, and Tone (GDELT), which classifies daily interactions as material or verbal, and cooperative or conflictual. Interactions between states were aggregated to the monthly level and first-differenced to account for the exponential growth in the number of records over time. Examples of dyadic time series are shown below, with shaded regions indicating the occurence of a MID. 

  \begin{center}
    \input{../graphics/timelines}
  \end{center}
 }


%%%%%%
  \headerbox{\bf Classification Trees}{name=estimations,column=1,row=0}{
%%%%%
  The figure below presents a graphical representation of the final classification tree, fit using the \texttt{rpart} package in \texttt{R}. The minimum complexity parameter, $\kappa$, was set to 0.0001. By ten-fold cross-validation, the optimal $\kappa$ was found to be 0.0016, which also minimizes error on the test data. This value was used to prune the tree shown here. The leaves of the tree are shaded according to whether war (MID hostilities $\geq$ 4 ) or peace is more likely, and the value at each leaf indicates the relative frequency of war in the training set. 

\vspace{-1.3in}
  \begin{center}
    \input{../graphics/tree}
  \end{center}
  
\vspace{-1.3in}
  Each table below presents several measures of accuracy for the classification tree compared to a null model (all predicted values are zero) and a logistic regression model using the same features as the tree. The test set covers 1992 to 1998 (\emph{n}=372,271) and the training set covers 1999-2001 (\emph{n}=213,218). The classification tree outperforms the other two models in all respects for the training data, and in most respects for the test data. 

  \begin{center}
  \begin{tabular}{l|ccc}
  \multicolumn{4}{c}{Training Data} \\
  Model & MSE & Precision & Recall \\
  \midrule
  Null & 0.0075 & 0.000 & 0.000 \\
  GLM & 0.0082 & 0.158 & 0.022 \\
  CART & 0.0067 & 0.702 & 0.192 
  \end{tabular}
  \end{center}

  \begin{center}
  \begin{tabular}{l|ccc}
  \multicolumn{4}{c}{Test Data} \\
  Model & MSE & Precision & Recall \\
  \midrule
  Null & 0.0066 & 0.000 & 0.000 \\
  GLM & 0.0079 & 0.142 & 0.038 \\
  CART & 0.0066 & 0.422 & 0.027 
  \end{tabular}
  \end{center}

   % \vspace{-.5in}
 }


%%%%%%
  \headerbox{\bf Results}{name=predictions,column=2,row=0}{
%%%%%
  Put in some maps.
 }

%%%%%%
  \headerbox{\bf Outlook}{name=conclusion,below=predictions,above=bottom,column=2}{
%%%%%
   \begin{itemize} \compresslist
      \item CART handles interactions between features better than GLM.
      \item Next steps: SMOTE for rare events, use spatial and network lages to account for interdependence between countries, compare to random forest.
      \item Automated methods can offer a quick and inexpensive first approximation of political indicators.
    \end{itemize}
%\vspace{0.3pt}
 }

\end{poster}
\end{document}